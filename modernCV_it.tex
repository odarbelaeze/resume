\documentclass[12pt,english]{moderncv}
\usepackage[T1]{fontenc}
\usepackage[utf8]{inputenc}
\usepackage[italian]{babel}
\setcounter{secnumdepth}{2}
\setcounter{tocdepth}{2}
\setlength{\parskip}{\medskipamount}
\setlength{\parindent}{0pt}
\usepackage{babel}

% \usepackage{bookmark}


\makeatletter
%%%%%%%%%%%%%%%%%%%%%%%%%%%%%% User specified LaTeX commands.
\moderncvtheme[green]{casual}
% possible themes are "classic" and "casual"
% optional argument are 'blue' (default), 'orange', 'red', 'green', 'grey' and 'roman' (for roman fonts, instead of sans serif fonts)

% required
\firstname{Oscar David\\}
% required
\familyname{Arbel\'aez Echeverri}

% optional, remove the line if not wanted
\title{Curriculum Vitae}

% optional
% \address{street and number}{postcode city}
% '\\' adds a line break
\address{Carrera 23  \#55A--35}{Manizales, Colombia}

% optional
% \phone{+57(6) 890 6069}
% optional
\mobile{+57 313 712 5660}
% optional
%\fax{+43(0)999 7777}
% optional
\email{odarbelaeze@unal.edu.co}
% optional
\extrainfo{%
    \href{https://github.com/odarbelaeze}{github.com/odarbelaeze}
}

% optional
% \photo[height]{name}
% 'height' is the height the picture is resized to
% 'name' is the name of the picture file
% \photo[64pt]{CV-image}

% optional
% \quote{"Everyone wants to be in the upper hand."}

\makeatother

\providecommand{\LyX}{\texorpdfstring%
  {L\kern-.1667em\lower.25em\hbox{Y}\kern-.125emX\@}
  {LyX}}

\newcommand*{\cvreference}[7][.25em]{%
    \cvitem[#1]{\color{color1}$\bullet$}{%
    \ifthenelse{\equal{#2}{}}{}{\textbf{#2}\newline}%
    \ifthenelse{\equal{#3}{}}{}{#3\newline}%
    \ifthenelse{\equal{#4}{}}{}{\addresssymbol~#4\newline}%
    \ifthenelse{\equal{#5}{}}{}{#5\newline}%
    \ifthenelse{\equal{#6}{}}{}{\emailsymbol~\href{mailto:#6}{\texttt{#6}}\newline}%
    \ifthenelse{\equal{#7}{}}{}{\phonesymbol~#7}
}}


\begin{document}

\maketitle

\begin{centering}
    Il codice sorgente ed i documenti di supporto per il curriculum vitae sono disponibili alla pagina \url{https://github.com/odarbelaeze/resume}. 
    Sono disponibile a fornire documenti aggiuntivi a quelli presenti sulla pagina web nel caso alcuni siano mancanti.
\end{centering}


\section{Istruzione}

\cventry{2012--2016}
        {Laurea magistrale in fisica}
        {Universidad Nacional de Colombia}
        {Manizales}
        {Colombia}
        {%
            Ho redatto la tesi magistrale intitolata ``Simulazioni con metodo Monte Carlo di bit patterned media applicate a dispositivi di memorizzazione magnetica dei dati''.
            Questo lavoro di tesi ha richiesto la comprensione della tecnologia dei dischi rigidi  allo stato dell'arte e della simulazione di materiali magnetici.
            Il risultato principale di questo lavoro \`e stato un modello accurato, sebbene relativamente semplice, dei bit patterned media. 
        }

\cventry{2008--2012}
        {Laurea triennale in ingegneria fisica}
        {Universidad Nacional de Colombia}
        {Manizales}
        {Colombia}
        {}


\section{Produzione scientifica}

\cventry{Marzo 2018}
        {Vegas: Software package for the atomistic simulation of magnetic materials}
        {Revista Mexicana de Física}
        {Sociedad Mexicana de Física}
        {}
        {DOI:\@ \href{https://doi.org/10.31349/RevMexFis.64.490}{10.31349/RevMexFis.64.490}}

\cventry{Marzo 2018}
        {Atomistic modelling of magnetic nano-granular thin films}
        {Physica E:\@ Low-dimensional Systems and Nanostructures}
        {Elsevier}
        {}
        {DOI:\@ \href{https://doi.org/10.1016/j.physe.2017.11.017}{10.1016/j.physe.2017.11.017}}

\cventry{Febbraio 2017}
        {%
            Spontaneous Perpendicular Anisotropy in Ultra-thin Ferromagnetic
            Films%
        }
        {Journal of Superconductivity and Novel Magnetism}
        {Springer}
        {}
        {DOI:\@ \href{https://doi.org/10.1007/s10948-017-4005-9}{10.1007/s10948--017--4005--9}}

\cventry{Febbraio 2017}
        {%
            Surface anisotropy and particle size influence on hysteresis loops
            in La2/3Ca1/3MnO3 nanoparticles: A simulation approach%
        }
        {Journal of Magnetism and Magnetic Materials}
        {Elsevier}
        {}
        {DOI:\@ \href{https://doi.org/10.1016/j.jmmm.2016.10.108}{10.1016/j.jmmm.2016.10.108}}

\cventry{Giugno 2016}
        {%
            Implementation details of a variational method to solve the time
            independent Schrödinger equation%
        }
        {Revista mexicana de física E}
        {Sociedad mexicana de física}
        {}
        {DOI:\@ \href{https://doi.org/10.31349/RevMexFis.64.490}{10.31349/RevMexFis.64.490}}

\cventry{Dicembre 2015}
        {Atomistic Simulation of static magnetic properties of bit patterned media}
        {Physica E}
        {Elsevier}
        {}
        {DOI:\@ \href{https://doi.org/10.1016/j.physe.2015.12.016}{10.1016/j.physe.2015.12.016}}

\cventry{Ottobre 2014}
        {%
            Simulation of magnetotransport properties of
            ferromagnetic/antiferromagnetic multilayers of manganites%
        }
        {Journal of Superconductivity and Novel Magnetism}
        {Springer}
        {}
        {DOI:\@ \href{https://doi.org/10.1007/s10948-014-2827-2}{10.1007/s10948--014--2827--2}}


\section{Esperienza lavorativa}

\cventry{Aprile 2016\\ ad oggi}
        {Full stack software engineer}
        {BairesDev}
        {}
        {}
        {%
            Ho fornito servizi di ``full stack contractor software engineering'' a
            \href{https://www.adroll.com/}{AdRoll} attraverso
            \href{http://www.bairesdev.com/}{BairesDev}.
            Le mie responsabilit\`a includevano scrivere e gestire il codice che permette a inserzionisti di far girare ``native ad-campaigns'' tramite AdRoll, sia in backend che in frontend.
        }

\cventry{Febbraio 2016\\--Maggio. 2017}
        {Professore a contratto}
        {Universidad Nacional de Colombia}
        {}
        {}
        {%
            Ho insegnato corsi presso il dipartimento di ``exact and natural sciences'' della Universidad Nacional de Colombia.
            Ho insegnato ``Simulation 1'' durante il primo trimestre del 2016, ``Informatica 3'' durante il secondo trimestre del 2016 e nel primo trimestre del 2017 ad ingegneria fisica.
        }

\cventry{Marzo 2015\\--Marzo 2016}
        {Consulente in tecnologia scientifica}
        {Universidad Nacional de Colombia}
        {}
        {}
        {%
            Ho fornito consulenza ed aiuto nei processi di acquisto di clusters di ridotta dimensione per simulazioni scientifiche per un gruppo di ricerca nel dipartimento di fisica della Universidad Nacional de Colombia. 
            Ho inoltre installato il cluster personalmente per ridurre i costi e per effettuare operazioni di gestione.
            Altre mansioni includevano insegnare come utilizzare il cluster e creare un sito web che permetta di migliorare la comunicazione tra i gruppi di ricerca e con l'esterno. 
            Il principale risultato \`e stato un utilizzo totale del cluster per parecchi mesi.
        }

\cventry{Luglio 2016\\--Agosto 2016}
        {Stagista estivo}
        {Computational Magnetism Group, University of York}
        {United Kindom}
        {}
        {%
            Ho avuto il compito di ultimare e testare il modulo CUDA per il codice per simulazioni scientifiche \textsc{Vampire}. Il codice \`e disponibile al sito \url{http://vampire.york.ac.uk/}.
        }

\cventry{Maggio 2015\\--Luglio 2015}
        {Stagista estivo}
        {Computational Magnetism Group, University of York}
        {United Kindom}
        {}
        {%
            Ho avuto il compito di guidare un ridotto team di sviluppo per introdurre una nuova funzione nel codice per simulazioni scientifiche \textsc{Vampire}.
            La principale difficolt\`a del progetto era integrare il modulo CUDA all'interno del codice principale implementato in seriale e in parallelo con MPI in C++.
            La feature \`e ancora in attesa di perfezionamento e approvazione: \url{https://github.com/richard-evans/vampire/pull/4}.
        }

\cventry{Maggio 2015\\--Luglio 2015}
        {Progetto: Web tools for smart selection of scientific articles applying network analysis}
        {Tree of Science}
        {}
        {}
        {%
            Ho rivestito una posizione manageriale dove ho dovuto guidare un ridotto team di sviluppo per rilasciare la versione per la produzione dello strumento ``Tree of Science'' ora presso \url{http://tos.manizales.unal.edu.co}.
        }

\cventry{Settembre 2014\\--Febbraio 2015}
        {Progetto: Web tools for smart selection of scientific articles applying network analysis}
        {Universidad Nacional de Colombia}
        {}
        {}
        {%
            Ho aiutato la ricerca e analisi della rete di algoritmi (algoritmi grafici) e indicatori di definizione. 
            Ho aiutato nel design e nella esecuzione di un workshop scientifico di analisi dati con python.
            Il prodotto principale era un prototipo operativo dello strumento ``Tree of science'' che ha lo scopo di fornire aiuto e guida sia a ricercatori giovani che esperti nella ricerca della letteratura scientifica necessaria per comprendere un certo ambito.
        }

\cventry{Febbraio 2013\\--Febbraio 2015}
        {Vincitore di borsa di studio}
        {Universidad Nacional de Colombia}
        {}
        {}
        {%
            Ho ricevuto una borsa di studio ed ho lavorato come giovane ricercatore nel gruppo accademico di ricerca PCM Computational Applications.
        }

\cventry{Agosto 2013\\--Dicembre 2016}
        {Consulente IT}
        {Fundación Alas de Cristal}
        {}
        {}
        {%
            Ho gestito il sito web della fondazione ed ho facilitato le relazioni della stessa con altri business partner nel settore IT.
        }

\cventry{Luglio 2013\\--Settembre 2013}
        {Web developing volunteer staff}
        {Universidad de Caldas}
        {}
        {}
        {%
            Ho sviluppato il sito web per la mostra sul libro del 2013 a Manizales \textit{(Feria del Libro de Manizales)}. Ho lavorato con un designer dell'universit\`a di Manizales per ottenere il miglior prodotto possibile utilizzando il minor numero di beni.
        }

\cventry{Gennaio 2013\\--Settembre 2013}
        {Assistente tecnico del chairman}
        {Universidad Nacional de Colombia}
        {}
        {}
        {%
            Ho aiutato ad organizzare la ``5th Latin-American Conference
            in Networked Electronic Media (LACNEM)''. 
            Ho prestato supporto al chairman della conferenza con lo sviluppo del sito della conferenza.
        }

\cventry{Ottobre 2012\\--Febbraio 2013}
        {Stagista}
        {Universidad Nacional de Colombia}
        {}
        {}
        {%
            Ho sviluppato codici scientifici, in particolar modo script in python e codici basati sull'algoritmo Monte Carlo in Fortran e C++. Ho usato questi script e codici per fare simulazioni atomistiche di materiali magnetici.
        }

\cventry{Agosto 2009\\--Maggio 2011}
        {Stagista}
        {Universidad Nacional de Colombia}
        {}
        {}
        {%
            Ho lavorato come assistente tecnico durante le esercitazioni pratiche di fisica meccanica, fisica dell'elettricit\`a e del magnetismo, biologia cellulare e fisica delle onde e delle oscillazioni presso la Universidad Nacional de Colombia.
        }


\section{Competenze linguistiche}

\cvlanguage{Spagnolo}
           {Madre lingua}
           {}

\cvlanguage{Inglese}
           {Molto fluente}
           {%
               MET, punteggio: 119 (B2 ascolto --- C1 comprensione scritta e grammatica)\\
               Berlitz, punteggio complessivo: B2\\
               Sono migliorato molto dall'ultimo esame
           }

\section{Competenze informatiche}

\cvcomputer{Sistemi operativi}            {Linux\\Windows\\MacOS}
           {Admin}         {AWS\\Docker\\Nginx}

\cvcomputer{Linguaggi di programmazione}   {Rust\\C/C++\\Python\\JavaScript}
           {Automazione}    {Autotools\\Dist Utils\\Unittest\\Gulp\\CMake}

\cvcomputer{Linguaggi di programmazione scientifica}    {NumPy\\SciPy\\Matplotlib\\Bokhe\\Pandas}
           {Software tipografici}    {pandoc\\markdown\\\LaTeX{}\\\LyX{}}


\cvcomputer{Software di web~design}    {React\\ES6\\HTML5\\sass}
           {Software di backend}       {Flask\\Django}


\section{Partecipazione a workshop e conferenze}

\cventry{Agosto\\2017}
        {Taller de Introducción a R y R Studio}
        {%
            Nodo Nacional de Bioinformática, Comunidad de Desarrolladoresde
            Sofware en Bioinformática, Centro de Ciencias Genómicas (CCG)
        }
        {Cuernavaca, Mexico}
        {30 Luglio --- 3 Agosto 2018}
        {(Istruttore)}

\cventry{Dicembre\\2017}
        {CODATA-RDA School on Data Science}
        {CODATA, Research Data Aliance, ICTP-SAIFR, Universidade Estadual Paulista}
        {São Paulo, Brasil}
        {30 Novembre --- 16 Dicembre 2017}
        {(Aiutante)}

\cventry{Luglio\\2017}
        {2017 CODATA-RDA Research Data Science Summer School}
        {CODATA, Research Data Aliance, Abdus Salam International Centre for Theoretical Physics}
        {Trieste, Italy}
        {10 --- 28 Luglio 2017}
        {(Partecipante)}

\cventry{Luglio\\2015}
        {20th International conference on magnetism}
        {Club español de magnetismo}
        {Barcelona, Spain}
        {5 --- 10 Luglio 2015}
        {(Partecipante con presentazione di un poster)}

\cventry{Maggio\\2015}
        {Workshop on accelerated high performance computing in computational sciences}
        {Abdus Salam International Centre for Theoretical Physics}
        {Trieste, Italy}
        {25 Maggio --- 5 Giugno 2015}
        {(Partecipante)}

\cventry{Settembre\\2014}
        {2nd Workshop on Statistical Physics}
        {Universidad de los Andes, Universidad Nacional de Colombia}
        {Bogotá, Colombia}
        {22 --- 26 Settembre 2014}
        {(Partecipante con presentazione orale)}

\cventry{Marzo\\2014}
        {LAMMPS Users and Developers Workshop and Symposium}
        {Abdus Salam International Centre for Theoretical Physics}
        {Trieste, Italia}
        {10 --- 21 Marzo 2014}
        {(Partecipante)}

\cventry{Marzo\\2014}
        {%
            Advanced Techniques for Scientific Programming and Management
            of Open Source Software Packages
        }
        {Abdus Salam International Centre for Theoretical Physics}
        {Trieste, Italia}
        {24 --- 28 Marzo 2014}
        {(Partecipante)}

\cventry{Settembre\\2013}
        {5th Latin American Workshop on Networked Electronic Media}
        {Universidad Nacional de Colombia}
        {Manizales, Colombia}
        {2 --- 4 Settembre 2013}
        {(Organizzatore)}


\cventry{Settembre\\2013}
        {1st Workshop on Applied Mathematics --- Region Cafetera}
        {Universidad Nacional de Colombia}
        {Manizales, Colombia}
        {2 --- 4 Settembre 2013}
        {(Organizzatore)}

\cventry{Settembre\\2012}
        {III Congreso Nacional de Ingeniería Física}
        {EAFIT University}
        {Medellín, Colombia}
        {10 --- 14 Settembre 2012}
        {(Partecipante con presentazione orale)}

\cventry{Febbraio\\2012}
        {Seminario Internacional de Ingeniería Biomédica (SIB2012)}
        {Universidad de los Andes}
        {Bogotá, Colombia}
        {10 --- 14 Settembre 2012}
        {(Partecipante)}

\cventry{Settembre\\2010}
        {II Congreso Nacional de Ingeniería Física}
        {Universidad Tecnol\'ogica de Pereira}
        {Pereira, Colombia}
        {6 --- 10 Settembre 2010}
        {(Partecipante)}

\cventry{Luglio\\2010}
        {IX Latin American Workshop on Magnetism, Magnetic Materials and their Applications}
        {Universidad Nacional de Colombia}
        {Manizales, Colombia}
        {26 --- 30 Luglio 2010}
        {(Partecipante)}

\cventry{Luglio\\2010}
        {IX Latin American School on Magnetism}
        {Universidad Nacional de Colombia}
        {Manizales, Colombia}
        {21 --- 24 Luglio 2010}
        {(Partecipante)}


\section{Formazione aggiuntiva}

\cventry{Primavera\\2018}
        {Software carpentry instructor certification}
        {}
        {Software carpentry}
        {Erin Becker}
        {9 Aprile 2018}

\cventry{Fine\\2013}
        {Machine Learning}
        {}
        {Corso}
        {Associate Professor Andrew Ng, Computer Science Department, Stanford University}
        {12 Novembre 2013}

\cventry{Fine\\2013}
        {HTML5 Game Development}
        {Creazione di applicazioni ``High Performance'' per il web}
        {Udacity}
        {Peter Lubbers and Colt McAnlis}
        {8 Novembre 2013}

\cventry{Estate\\2013}
        {Curso Introductorio a la Simulación Monte Carlo}
        {Corso introduttorio alle simulationi con metodo Monte Carlo}
        {Universidad Católica de Pereira}
        {}
        {11 Settembre 2013}

\cventry{Primavera\\2013}
        {Computer Science 101}
        {}
        {Corso}
        {Nick Parlante, Lecturer in Computer Science, Stanford University}
        {13 Maggio 2013}

\cventry{Fine\\2011}
        {Estructura del lenguaje de programación C++ (Nivel II)}
        {Struttura del linguaggio di programmazione C++ (Secondo livello)}
        {Servicio Nacional de Aprendizaje, SENA}
        {}
        {9 Gennaio 2012}

\cventry{Estate\\2011}
        {Metrología: Patrones, instrumentos y tolerancias}
        {Metrologia: Schemi, strumenti e tolleranze}
        {Servicio Nacional de Aprendizaje, SENA}
        {}
        {11 Novembre 2011}

\cventry{Primavera\\2011}
        {Estructura del lenguaje de programación C++ (Nivel I)}
        {Struttura del linguaggio di programmazione C++ (Primo livello)}
        {Servicio Nacional de Aprendizaje, SENA}
        {}
        {18 Marzo 2011}


\section{Riconoscimenti}

\cventry{2016}
        {Distinción meritoria en tesis de posgrado}
        {Distinzione con merito per tesi magistrale}
        {Universidad Nacional de Colombia}
        {}
        {(premiato dal consiglio di facolt\`a dopo la raccomandazione del relatore esterno della tesi)}

\cventry{2013}
        {Beca Grado de Honor de Pregrado}
        {Borsa di studio per laurea con lode}
        {Universidad Nacional de Colombia}
        {}
        {}

\cventry{2012}
        {Grado de Honor}
        {Laurea con lode}
        {Universidad Nacional de Colombia}
        {}
        {}


\cventry{2009 --- 2011}
        {Exención de pago de derechos académicos}
        {Esenzione dal pagamento delle tasse universitarie}
        {Universidad Nacional de Colombia}
        {}
        {(ottenuto due volte nel 2009, nel 2010 e 2012)}


\cventry{2007}
        {Decreto de Honores No. 0787}
        {Decreto di Onore No. 0787}
        {Gobernacíon de Caldas}
        {6 Dicembre 2007}
        {(riconoscimento annuale ai dieci migliori studenti, insegnanti e istituzioni d'insegnamento di Caldas)}


\cventry{2007}
        {Orden José Joaquin Montes}
        {Ordine di José Joaquin Montes}
        {Manzanares (Caldas, Colombia) Majorship}
        {1 Dicembre 2007}
        {(riconoscimento annuale al miglior studente diplomato tra le scuole superiori di Manzanares)}


\section{Referenze}


\cvreference%
        {Prof.~Elisabeth Restrepo Parra}
        {Departamento de Ciencias Exactas y Naturales}
        {Universidad Nacional de Colombia}
        {Carrera 27 \# 64--60, Manizales, Colombia}
        {erestrepopa@unal.edu.co}
        {+57 321 700 4351}


\cvreference%
        {Ivan Girotto}
        {Information \& Communication Technology Section}
        {The Abdus Salam-International Centre for Theoretical Physics}
        {Strada Costiera, 11 --- I--34151, Trieste, Italy}
        {igirotto@ictp.it}
        {+39 040 2240 484}


\cvreference%
        {Dr.~Axel Kohlmeyer}
        {Institute for Computational Molecular Science (035--07)}
        {College of Science and Technology, Temple University}
        {Philadelphia, PA 19122}
        {akohlmey@gmail.com}
        {+1 215 204 4218}


\cvreference%
        {Dr.~Richard F. L. Evans}
        {The Department of Physics}
        {University of York}
        {YO10 5DD, York, UK}
        {richard.evans@york.ac.uk}
        {+44 (0)1904 322822}


\cvreference%
        {Robert Quick}
        {Deputy Director Science Gateways Research Center}
        {Indiana University}
        {}
        {rquick@iu.edu}
        {+1 317 274 5260}


\end{document}
