%% LyX 2.0.6 created this file.  For more info, see http://www.lyx.org/.%% Do not edit unless you really know what you are doing.
\documentclass[12pt,english]{moderncv}
\usepackage[T1]{fontenc}
\usepackage[utf8]{inputenc}
\setcounter{secnumdepth}{2}
\setcounter{tocdepth}{2}
\setlength{\parskip}{\medskipamount}
\setlength{\parindent}{0pt}
\usepackage{babel}

% \usepackage{bookmark}


\makeatletter
%%%%%%%%%%%%%%%%%%%%%%%%%%%%%% User specified LaTeX commands.
\moderncvtheme[green]{casual}
% possible themes are "classic" and "casual"
% optional argument are 'blue' (default), 'orange', 'red', 'green', 'grey' and 'roman' (for roman fonts, instead of sans serif fonts)

% required
\firstname{Oscar David\\}
% required
\familyname{Arbel\'aez Echeverri}

% optional, remove the line if not wanted
\title{Curriculum Vitae}

% optional
% \address{street and number}{postcode city}
% '\\' adds a line break
\address{Calle 62A  \#18--97}{Manizales, Colombia}

% optional
% \phone{+57(6) 890 6069}
% optional
\mobile{+57(313) 712 5660}
% optional
%\fax{+43(0)999 7777}
% optional
\email{odarbelaeze@gmail.com}
% optional
\extrainfo{%
    \href{https://github.com/odarbelaeze}{github.com/odarbelaeze}
}

\makeatother

\providecommand{\LyX}{\texorpdfstring%
  {L\kern-.1667em\lower.25em\hbox{Y}\kern-.125emX\@}
  {LyX}}

\newcommand*{\cvreference}[7][.25em]{%
    \cvitem[#1]{\color{color1}$\bullet$}{%
    \ifthenelse{\equal{#2}{}}{}{\textbf{#2}\newline}%
    \ifthenelse{\equal{#3}{}}{}{#3\newline}%
    \ifthenelse{\equal{#4}{}}{}{\addresssymbol~#4\newline}%
    \ifthenelse{\equal{#5}{}}{}{#5\newline}%
    \ifthenelse{\equal{#6}{}}{}{\emailsymbol~\href{mailto:#6}{\texttt{#6}}\newline}%
    \ifthenelse{\equal{#7}{}}{}{\phonesymbol~#7}
}}


\begin{document}

\maketitle

\begin{centering}
    Las fuentes para este documento estan disponibles en
    \url{https://github.com/odarbelaeze/resume}, pueden preguntar por
    documentos de soporte adicionales si hace falta.
\end{centering}

\section{Educación}

\cventry{2012--actual}
        {Maestría en ciencias --- Física}
        {Universidad Nacional de Colombia}
        {Manizales}
        {Colombia}
        {}

\cventry{2008--2012}
        {Pregrado en ingeniería física}
        {Universidad Nacional de Colombia}
        {Manizales}
        {Colombia}
        {}

\section{Artículos~científicos}

\cventry{March 2018}
        {Vegas: Software package for the atomistic simulation of magnetic materials}
        {Revista Mexicana de Física}
        {Sociedad Mexicana de Física}
        {}
        {DOI:\@ \href{https://doi.org/10.31349/RevMexFis.64.490}{10.31349/RevMexFis.64.490}}

\cventry{March 2018}
        {Atomistic modelling of magnetic nano-granular thin films}
        {Physica E:\@ Low-dimensional Systems and Nanostructures}
        {Elsevier}
        {}
        {DOI:\@ \href{https://doi.org/10.1016/j.physe.2017.11.017}{10.1016/j.physe.2017.11.017}}

\cventry{February 2017}
        {%
            Spontaneous Perpendicular Anisotropy in Ultra-thin Ferromagnetic
            Films
        }
        {Journal of Superconductivity and Novel Magnetism}
        {Springer}
        {}
        {DOI:\@ 10.1007/s10948--017--4005--9}

\cventry{February 2017}
        {%
            Surface anisotropy and particle size influence on hysteresis loops
            in La2/3Ca1/3MnO3 nanoparticles: A simulation approach
        }
        {Journal of Magnetism and Magnetic Materials}
        {Elsevier}
        {}
        {DOI:\@ 10.1016/j.jmmm.2016.10.108}

\cventry{June 2016}
        {%
            Implementation details of a variational method to solve the time
            independent Schrödinger equation.
        }
        {Revista mexicana de física E}
        {Sociedad mexicana de física}
        {}
        {}

\cventry{December 2015}
        {Atomistic Simulation of static magnetic properties of bit patterned media}
        {Physica E}
        {Elsevier}
        {}
        {DOI:\@ 10.1016/j.physe.2015.12.016}

\cventry{October 2014}
        {%
            Simulation of magnetotransport properties of
            ferromagnetic/antiferromagnetic multilayers of manganites
        }
        {Journal of Superconductivity and Novel Magnetism}
        {Springer}
        {}
        {DOI:\@ 10.1007/s10948--014--2827--2}


\section{Experiencia}

\cventry{Nov. 2019\\--actual}
        {Ingeniero de Software --- Full Stack}
        {BairesDev}
        {}
        {}
        {%
            En este trabajo mi responsabilidad se centra en proveer servicios
            de desarrollo de software full stack a la compañía
            \href{https://www.adroll.com/}{AdRoll} a través de
            \href{http://www.bairesdev.com/}{BairesDev}.
        }

\cventry{Jul. 2019\\--Oct. 2019}
        {Contratista en Desarrollo de Software}
        {Acaris Labs S.A.S}
        {}
        {}
        {%
            En este contrato mis responsabilidades se centraban en el
            desarrollo de una integracion entre el software del cliente
            \href{https://www.playvox.com/}{PlayVox} y la plataforma
            \url{https://www.wixanswers.com/}.
        }

\cventry{Feb. 2019\\--Nov. 2019}
        {Asistente Doctoral}
        {EPFL}
        {Suiza}
        {}
        {%
            En este trabajo mis responsabilidades consistían en colaborar con
            el trabajo del grupo de investigación
            \href{http://theossrv1.epfl.ch/}{Theos} mientras realizaba mis
            actividades de formación doctoral en materiales de los materiales.
            \textit{(Esta actividad y la formació doctoral se vieron
            interrumpidas por problemas de salud)}
        }

\cventry{Abr. 2016\\--Feb. 2019}
        {Ingeniero de Software --- Full Stack}
        {BairesDev}
        {}
        {}
        {%
            En este trabajo mi responsabilidad era proveer servicios de desarrollo
            de software full stack a la compañía
            \href{https://www.adroll.com/}{AdRoll} a través de
            \href{http://www.bairesdev.com/}{BairesDev}.
        }

\cventry{Feb. 2016\\--May. 2017}
        {Profesor Universitario}
        {Universidad Nacional de Colombia}
        {}
        {}
        {%
            Orienté las asignaturas: Simulación 1 para Ingeniería Física e
            Informática 3 para ingeniería física.
        }

\cventry{Mar. 2015\\--Mar. 2016}
        {Consultor científico en tecnología}
        {Universidad Nacional de Colombia}
        {}
        {}
        {%
            Principalmente, mis responsabilidades se centraban en auyudar en
            el proceso de compra de un pequeño cluster de computacion de alto
            rendimiento.
        }

\cventry{Jul. 2016\\--Ago. 2016}
        {Pasante}
        {Computational Magnetism Group, University of York}
        {Reino Unido}
        {}
        {%
            En esta pasantía tuve la responsabilidad de darle los toques finales
            y realizar pruebas de rendimiento al módulo de soporte CUDA del paquete
            de simulación \textsc{Vampire}.
            \url{http://vampire.york.ac.uk/}
        }

\cventry{May. 2015\\--Jul. 2015}
        {Pasante}
        {Computational Magnetism Group, University of York}
        {Reino Unido}
        {}
        {%
            En esta pasantía tuve la responsabilidad de liderar un pequeño
            equipo de desarrollo de software para añadir una nueva característica
            al paquiete de simulación \textsc{Vampire}. El reto de este proyecto
            era integrar el soporte para CUDA en el código del paquete.
            \url{https://vampire.york.ac.uk}
        }

\cventry{Sept. 2014\\--Jul. 2015}
        {%
            Proyecto: aplicación web para selección inteligente de
            artículos científicos
        }
        {Universidad Nacional de Colombia}
        {}
        {}
        {%
            Ayuda en el análisis, investigación de algoritmos
            de redes (algoritmos de grafos) y definición de
            indicadores; diseñar y entregar un taller de análisis
            de datos científicos usando python.
        }

\cventry{Feb. 2013\\--Feb. 2015}
        {Becario}
        {Universidad Nacional de Colombia}
        {}
        {}
        {%
            Becario e investigador en el grupo de trabajo académico
            PCM Computational Applications.
        }

\cventry{Ago. 2013\\--Dec. 2016}
        {Desarrollador web}
        {Fundación Alas de Cristal}
        {}
        {}
        {Desarrollador de software y consultor en tecnología.}

\cventry{Jul. 2013\\--Sept. 2013}
        {Desarrollador de software}
        {Universidad de Caldas}
        {}
        {}
        {%
            Desarrollador del sitio web
            \href{http://ferialdellibromz.com}{http://feriadellibromz.com}
            (archivado)
        }

\cventry{Ene. 2013\\--Sept. 2013}
        {Asistente técnico}
        {Universidad Nacional de Colombia}
        {}
        {}
        {%
            Organización del \textit{5th Latin American Conference
            in Networked Electronic Media}
            \href{http://lacnem.org}{http://lacnem.org}
        }

\cventry{Oct. 2012\\--Feb. 2013}
        {Estudiante auxiliar}
        {Universidad Nacional de Colombia}
        {}
        {}
        {%
            Desarrollo de software científico, implementación
            de algoritmos Monte Carlo para la simulación de
            propiedades físicas de materiales.
        }

\cventry{Ago. 2009\\--May 2011}
        {Monitor académico}
        {Universidad Nacional de Colombia}
        {}
        {}
        {%
            Monitor académico en las prácticas de los cursos de física
            mecánica, física de electricidad y magnetísmo, biología celular
            y física de oscilaciones y ondas.
        }


\section{Idiomas}

\cvlanguage{Español}
           {Nativo}
           {}

\cvlanguage{Inglés}
           {Muy bueno}
           {MET, \\ Calificación: 119 (B2 Escucha --- C1 Lectura y gramática)}

\cvlanguage{Inglés}
           {Muy bueno}
           {Berlitz, \\ Calificación acumulada: B2}


\section{Habilidades~computacionales}

\cvcomputer{OS}            {Linux\\Windows}
           {Admin}         {Apache\\WSGI}

\cvcomputer{Programación}  {Rust\\C/C++\\Python\\JavaScript\\PHP}
           {Automat.}      {Autotools\\distutils\\Unittest\\Gulp\\CMake}

\cvcomputer{Científico}    {Octave\\SciDavis\\GnuPlot\\scipy stack}
           {Typografía}    {pandoc\\markdown\\\LaTeX{}\\\LyX{}}


\cvcomputer{Diseño~Web}    {HTML5\\CSS3}
           {Backend}       {Django\\NodeJS\\Laravel}


\section{Participación~en~eventos}

\cventry{Agosto\\2017}
        {Taller de Introducción a R y R Studio}
        {%
            Nodo Nacional de Bioinformática, Comunidad de Desarrolladoresde
            Sofware en Bioinformática, Centro de Ciencias Genómicas (CCG)
        }
        {Cuernavaca, México}
        {Jul 30 --- Ago 3, 2018}
        {(Asistente profesoral)}

\cventry{Diciembre \\2017}
        {CODATA-RDA School on Data Science}
        {CODATA, Research Data Aliance, ICTP-SAIFR, Universidade Estadual Paulista}
        {São Paulo, Brasil}
        {Nov 30 --- Dic 16, 2017}
        {(Asistente profesoral)}

\cventry{Julio\\2017}
        {2017 CODATA-RDA Research Data Science Summer School}
        {CODATA, Research Data Aliance, Abdus Salam International Centre for Theoretical Physics}
        {Trieste, Italia}
        {Julio 10 --- 28, 2017}
        {(Participante)}

\cventry{Julio\\2015}
        {20th International conference on magnetism}
        {Club español de magnetismo}
        {Barcelona, España}
        {Julio 5 --- 10, 2015}
        {(Poster)}

\cventry{Mayo\\2015}
        {Workshop on accelerated high performance computing in computational sciences}
        {Abdus Salam International Centre for Theoretical Physics}
        {Trieste, Italia}
        {May 25 --- Jun 5, 2015}
        {(Participante)}

\cventry{Septiembre\\2014}
        {2nd Workshop on Statistical Physics}
        {Universidad de los Andes, Universidad Nacional de Colombia}
        {Bogotá, Colombia}
        {Septiembre 22 --- 26, 2014}
        {(Ponente)}

\cventry{Marzo\\2014}
        {LAMMPS Users and Developers Workshop and Symposium}
        {Abdus Salam International Centre for Theoretical Physics}
        {Trieste, Italia}
        {Marzo 10 --- 21, 2014}
        {(Participante)}

\cventry{Marzo\\2014}
        {%
            Advanced Techniques for Scientific Programming and Management
            of Open Source Software Packages
        }
        {Abdus Salam International Centre for Theoretical Physics}
        {Trieste, Italia}
        {Marzo 24 --- 28, 2014}
        {(Participante)}

\cventry{Septiembre\\2013}
        {5th Latin American Workshop on Networked Electronic Media}
        {Universidad Nacional de Colombia}
        {Manizales, Colombia}
        {Septiembre 2 --- 4, 2013}
        {(Organizador)}


\cventry{Septiembre\\2013}
        {1st Workshop on Applied Mathematics --- Region Cafetera}
        {Universidad Nacional de Colombia}
        {Manizales, Colombia}
        {Septiembre 2 --- 4, 2013}
        {(Organizador)}

\cventry{Septiembre\\2012}
        {III Congreso Nacional de Ingeniería Física}
        {Universidad EAFIT}
        {Medellín, Colombia}
        {Septiembre 10 --- 14, 2012}
        {(Ponente)}

\cventry{Febrero\\2012}
        {Seminario Internacional de Ingeniería Biomédica (SIB2012)}
        {Universidad de los Andes}
        {Bogotá, Colombia}
        {Febrero, 2012}
        {(Participante)}

\cventry{Septiembre\\2010}
        {II Congreso Nacional de Ingeniería Física}
        {Universidad Tecnológica de Pereira}
        {Pereira, Colombia}
        {Septiembre 6 --- 10, 2010}
        {(Participante)}

\cventry{Julio\\2010}
        {%
            IX Latin American Workshop on Magnetism, Magnetic
            Materials and their Applications
        }
        {Universidad Nacional de Colombia}
        {Manizales, Colombia}
        {Julio 26 --- 30, 2010}
        {(Participante)}

\cventry{Julio\\2010}
        {IX Latin American School on Magnetism}
        {Universidad Nacional de Colombia}
        {Manizales, Colombia}
        {Julio 21 --- 24, 2010}
        {(Participante)}


\section{Complementary~Education}

\cventry{Primavera\\2018}
        {Software carpentry instructor certification}
        {Certificación como instructor de software carpentry}
        {Software carpentry}
        {Erin Becker}
        {Abril 9, 2018}

\cventry{Otoño\\2013}
        {Machine Learning}
        {}
        {Coursera}
        {Profesor Asociado Andrew Ng, Computer Science Department, Stanford University}
        {Noviembre 12, 2013}

\cventry{Otoño\\2013}
        {HTML5 Game Development}
        {Building High Performance Web Applications}
        {Udacity}
        {Peter Lubbers and Colt McAnlis}
        {Noviembre 8, 2013}

\cventry{Verano\\2013}
        {Curso Introductorio a la Simulación Monte Carlo}
        {}
        {Universidad Católica de Pereira}
        {}
        {Septiembre 11--11, 2013}

\cventry{Primavera\\2013}
        {Computer Science 101}
        {}
        {Coursera}
        {Nick Parlante, Profesor en Computer Science, Stanford University}
        {Mayo 13, 2013}

\cventry{Otoño\\2011}
        {Estructura del lenguaje de programación C++ (Nivel II)}
        {}
        {Servicio Nacional de Aprendizaje, SENA}
        {}
        {Noviembre 23, 2011}

\cventry{Verano\\2011}
        {Metrología: Patrones, instrumentos y tolerancias}
        {}
        {Servicio Nacional de Aprendizaje, SENA}
        {}
        {Noviembre 23, 2011}

\cventry{Primavera\\2011}
        {Estructura del lenguaje de programación C++ (Nivel I)}
        {}
        {Servicio Nacional de Aprendizaje, SENA}
        {}
        {Marzo 18, 2011}


\section{Reconocimientos}

\cventry{2013}
        {Beca Grado de Honor de Pregrado}
        {}
        {Universidad Nacional de Colombia}
        {}
        {}

\cventry{2012}
        {Grado de Honor}
        {}
        {Universidad Nacional de Colombia}
        {}
        {}


\cventry{2009 --- 2011}
        {Exención de pago de derechos académicos}
        {}
        {Universidad Nacional de Colombia}
        {}
        {(obtenida dos veces en 2009, así como en 2010 y 2012)}


\cventry{2007}
        {Decreto de Honores No. 0787}
        {}
        {Gobernacíon de Caldas}
        {Diciembre 6, 2007}
        {%
            (obtenida anualmente por los 10 mejores estudiantes,
            profesores e instituciones educativas en Caldas,
            Colombia)
        }


\cventry{2007}
        {Orden José Joaquin Montes}
        {}
        {Alcaldía de Manzanares (Caldas, Colombia)}
        {Diciembre 1st, 2007}
        {%
            (earned each year by the best student graduating from
            high school in Manzanares)
        }


\end{document}
